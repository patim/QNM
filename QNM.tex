\documentclass[11pt]{article}
\usepackage[letterpaper,margin=0.5in,nohead,nofoot]{geometry}
\usepackage{color}
\usepackage{amsfonts}
\usepackage{amssymb}
\usepackage{amsmath}
\usepackage{setspace}


\everymath{\displaystyle}
\onehalfspacing

%\newcommand{\bra}[1]{\langle #1|}
%\newcommand{\ket}[1]{|#1\rangle}
\newcommand{\braket}[2]{\langle #1|#2\rangle}
\mathchardef\minus = "002D

\newcommand{\swY}[4][]{{}_{{}_{#2}}\!Y^{#1}_{#3}(#4)}
\newcommand{\swSH}[5][]{{}_{{}_{#2}}S^{#1}_{#3}(#4;#5)}
\newcommand{\swS}[5][]{{}_{{}_{#2}}S^{#1}_{#3}(#4;#5)}
\newcommand{\scA}[4][]{{}_{{}_{#2}}A^{#1}_{#3}(#4)}
\newcommand{\YSH}[3][]{\mathcal{A}^{#1}_{#2}(#3)}

\begin{document}
\noindent{\large\bf Quasi-normal Mode Decomposition for the ring-down} \\

\noindent
Teukolsky general field solution:
\begin{equation}
\psi(t,r,\theta,\phi) = \frac{1}{2\pi} \int {e^{-i\omega t} \sum_{\ell,m} \swSH{s}{\ell{m}}{\theta,\phi}{a\omega}R_{\ell{m}}(r) d\omega },
\end{equation}
where the summation over $\ell$ and $m$ is always confined to $|s|<
\ell < \infty$ and $|m| \leq \ell$. To represent outgoing
gravitational waves, we use $s=-2$ and $\psi = \rho^{-4}\Psi_4$, with
$\Psi_4$ being the Weyl scalar that behaves at large distances like
outgoing gravitational waves and $\rho\equiv-1/(r-ia\cos\theta)$.
Asymptotically, $R_{lm}(r)=r^3e^{i\omega r_*}$ ($r_*$ is the
''tortoise'' radial coordinate), $\rho=1/r$, and we find
\begin{equation}
\Psi_4 \sim \frac1{r}\frac{1}{2\pi} \int {e^{-i\omega (t-r_*)} \sum_{\ell,m} \swSH{\minus 2}{\ell{m}}{\theta,\phi}{a\omega} d\omega }.
\end{equation}
Making both sides dimensionless, we take as our general asymptotic expansion for outgoing gravitational waves:
\begin{equation}
rM\Psi_4 = \frac{M}{2\pi} \int {e^{-i\omega (t-r_*)} \sum_{\ell,m} \swSH{\minus 2}{\ell{m}}{\theta,\phi}{a\omega} d\omega }.
\end{equation}

To represent the ring-down of a Kerr black hole, we replace the
Fourier integral with a QNM expansion (a careful assumption, because
QNMs are not a complete set).  For each mode $(\ell,m)$, there is are
two infinite sets of QNMs labeled by $n$: $\bar\omega_{\ell{m}n}$ and
$\bar\omega^\prime_{\ell{m}n}$.  The $n=0$ modes represent the
dominant QNM for each $(\ell,m)$, and the higher values of $n$
represent overtones.  When we replace the integral with a sum over all QNM, we must include both sets of modes:
\begin{equation}
rM\Psi_4 = \sum_{\ell{m}n}\left\{C_{\ell{m}n}e^{-i\bar\omega_{\ell{m}n} (t-r_*)}
           \swSH{\minus 2}{\ell{m}}{\theta,\phi}{a\bar\omega_{\ell{m}n}} 
           + C^\prime_{\ell{m}n}e^{-i\bar\omega^\prime_{\ell{m}n} (t-r_*)}
           \swSH{\minus 2}{\ell{m}}{\theta,\phi}{a\bar\omega^\prime_{\ell{m}n}}
           \right\}, 
\end{equation}
where $C_{\ell{m}n}$ and $C^\prime_{\ell{m}n}$ are the arbitrary
dimensionless complex coefficients of the QNM expansion.

The two sets of QNMs for Kerr are closely related to each other.  The $\bar\omega$ correspond to ``positive frequency'' modes and the $\bar\omega^\prime$ to ``negative frequency'' modes in the following sense.  Let us define:
\begin{align}
    \bar\omega_{\ell{m}n} &\equiv 
    \omega_{\ell{m}n} - \frac{i}{\tau_{\ell{m}n}}
      &;\quad a\bar\omega_{\ell{m}n}\equiv c_{\ell{m}n}
    &\quad:\quad\mbox{and}\quad \omega_{\ell{m}n} \ge 0, \\
    \bar\omega^\prime_{\ell{m}n} &\equiv 
    \omega^\prime_{\ell{m}n} - \frac{i}{\tau^\prime_{\ell{m}n}}
      &;\quad a\bar\omega^\prime_{\ell{m}n}\equiv c^\prime_{\ell{m}n}
    &\quad:\quad\mbox{and}\quad \omega^\prime_{\ell{m}n} \le 0.
\end{align}
They are further related by the fact that:
\begin{align}
\omega^\prime_{\ell{m}n} = -\omega_{\ell(-m)n} \\
\tau^\prime_{\ell{m}n} = \tau_{\ell(-m)n}.
\end{align} 
In terms of these definitions, and assuming we evaluate the expansion at fixed $r$ and $r_*$, we find:
\begin{equation}
rM\Psi_4 = \sum_{\ell{m}n} \left\{ C_{\ell{m}n} e^{-i\omega_{\ell{m}n}t}e^{-t/\tau_{\ell{m}n}} \swSH{\minus 2}{\ell{m}}{\theta,\phi}{c_{\ell{m}n}} + C^\prime_{\ell{m}n} e^{i\omega_{\ell(-m)n}}e^{-t/\tau_{\ell(-m)n}} \swSH{\minus 2}{\ell{m}}{\theta,\phi}{c^\prime_{\ell{m}n}} \right\},
\end{equation}
where we have absorbed constant factors involving $e^{r_*}$ into the
expansion coefficients.

Using the fact that the sum over $m$ extends from $-\ell\cdots\ell$, we can trivially rewrite the expansion as:
\begin{equation}
rM\Psi_4 = \sum_{\ell{m}n} \left\{ C_{\ell{m}n} e^{-i\omega_{\ell{m}n}t}e^{-t/\tau_{\ell{m}n}} \swSH{\minus 2}{\ell{m}}{\theta,\phi}{c_{\ell{m}n}} + C^\prime_{\ell(-m)n} e^{i\omega_{\ell{m}n}}e^{-t/\tau_{\ell{m}n}} \swSH{\minus 2}{\ell(-m)}{\theta,\phi}{c^\prime_{\ell(-m)n}} \right\}.
\end{equation}
Furthermore, because
\begin{equation}
c^\prime_{\ell(-m)n} = a\left(\omega^\prime_{\ell(-m)n} 
               - \frac{i}{\tau^\prime_{\ell(-m)n}}\right) =
               -a\left(\omega_{\ell{m}n} 
               - \frac{i}{\tau_{\ell{m}n}}\right)^* = -c^*_{\ell{m}n},
\end{equation}
and by the symmetries of the spin-weighted spheroidal harmonics given in Eqns~(\ref{eqn:swSHminuss}) and (\ref{eqn:swSHminusm}), we find
\begin{equation}
\swSH{\minus 2}{\ell(-m)}{\theta,\phi}{c^\prime_{\ell(-m)n}} = (-1)^l \swSH[*]{\minus 2}{\ell{m}}{\pi-\theta,\phi}{c_{\ell{m}n}},
\end{equation}

\noindent
Plugging those in, we get
\begin{equation}
\nonumber rM\Psi_4 = \sum_{\ell{m}n} \left\{C_{\ell{m}n} e^{-i\omega_{\ell{m}n}t}e^{-t/\tau_{\ell{m}n}} \swSH{\minus 2}{\ell{m}}{\theta,\phi}{c_{\ell{m}n}}
  + (-1)^\ell C^\prime_{\ell(-m)n} e^{i\omega_{\ell{m}n}t}e^{-t/\tau_{\ell{m}n}} \swSH[*]{\minus 2}{\ell{m}}{\pi-\theta,\phi}{c_{\ell{m}n}} \right\}.
\end{equation}
Finally, we can rewrite the two complex expansion coefficients as
\begin{align}
  C_{\ell{m}n} &\equiv A_{\ell{m}n}e^{i\phi_{\ell{m}n}}, \\
  C^\prime_{\ell(-m)n} &\equiv (-1)^\ell A^\prime_{\ell{m}n}e^{i\phi^\prime_{\ell{m}n}},
\end{align}
where $A_{\ell{m}n}$, $A^\prime_{\ell{m}n}$, $\phi_{\ell{m}n}$ and $\phi^\prime_{\ell{m}n}$ are real parameters.  The resulting equation is:
\begin{align} \label{eqn:Psi4SH:expan}
\nonumber rM\Psi_4 &= \sum_{\ell{m}n} \Bigl\{A_{\ell{m}n} e^{-i\omega_{\ell{m}n}t+i\phi_{\ell{m}n}}e^{-t/\tau_{\ell{m}n}} \swSH{\minus 2}{\ell{m}}{\theta,\phi}{c_{\ell{m}n}} \nonumber\\ & {} \hspace{0.5in}
  + A^\prime_{\ell{m}n} e^{i\omega_{\ell{m}n}t+i\phi^\prime_{\ell{m}n}}e^{-t/\tau_{\ell{m}n}} \swSH[*]{\minus 2}{\ell{m}}{\pi-\theta,\phi}{c_{\ell{m}n}} \Bigr\}.
\end{align}

\noindent\\
On the other hand, the Weyl scalar can also be decomposed using spin-weight $\minus 2$ spherical harmonics.  Again, assuming fixed $r$ and $r^*$:
\begin{equation}\label{eqn:Psi4Y:expan}
rM\Psi_4 = \sum_{\ell{m}} C_{\ell{m}}(t) \swY{\minus 2}{\ell{m}}{\theta,\phi},
\end{equation} 
The next step is to relate complex amplitudes $C_{\ell{m}}$ to the parameters of the ring-down signal (\ref{eqn:Psi4SH:expan}).  To accomplish this, we first expand the spheroidal harmonics in terms of spherical harmonics:
\begin{equation} \label{eqn:swSH:expan}
\swSH{\minus 2}{\ell{m}n}{\theta,\phi}{c} = \sum_{\ell^\prime} \YSH{\ell^\prime\ell{m}n}{c} \swY{\minus 2}{\ell^\prime{m}}{\theta,\phi}.
\end{equation}
Using Eqns~(\ref{eqn:swYminuss}), (\ref{eqn:swYminusm}), (\ref{eqn:swSHminuss}), and (\ref{eqn:swSHminusm}), we also have
\begin{equation} \label{eqn:swSconj:expan}
\swSH[*]{\minus 2}{\ell{m}n}{\pi-\theta,\phi}{c} = \sum_{\ell^\prime} (-1)^{\ell^\prime}\YSH[*]{\ell^\prime\ell{m}n}{c} \swY{\minus 2}{\ell^\prime(-m)}{\theta,\phi}.
\end{equation}
(Note: $\YSH[*]{\ell^\prime\ell{m}n}{c} = \YSH{\ell^\prime\ell(-m)n}{-c^*}$.)

So, equating Eqs.~(\ref{eqn:Psi4SH:expan}) and (\ref{eqn:Psi4Y:expan}), 
\begin{equation} \label{eqn:equate:expan}
\begin{aligned}
\sum_{\ell{m}} C_{\ell{m}}(t) \swY{\minus 2}{\ell{m}}{\theta,\phi} & = \sum_{\ell{m}n} \Big\{ A_{\ell{m}n} e^{-i\omega_{\ell{m}n}t+i\phi_{\ell{m}n}}e^{-t/\tau_{\ell{m}n}} \swSH{\minus 2}{\ell{m}}{\theta,\phi}{c_{\ell{m}n}}\\
& + A^\prime_{\ell{m}n} e^{i\omega_{\ell{m}n}t+i\phi^\prime_{\ell{m}n}}e^{-t/\tau_{\ell{m}n}} \swSH[*]{\minus 2}{\ell{m}}{\pi-\theta,\phi}{c_{\ell{m}n}} \Big\},
\end{aligned}
\end{equation}
and using the usual definition for the inner product $\braket{f(\theta, \phi)}{g(\theta, \phi)} \equiv \int{f^{*}(\theta, \phi) g(\theta, \phi) d\Omega}$ along with the orthonormality of the spin-weighted spherical harmonics, Eq,~(\ref{eqn:equate:expan}) becomes
\begin{equation} \label{clm:eq2}
\begin{aligned}
C_{\ell^\prime{m^\prime}}(t) &= \sum_{\ell{m}n} \Big\{ 
   A_{\ell{m}n} e^{-i\omega_{\ell{m}n}t+i\phi_{\ell{m}n}}e^{-t/\tau_{\ell{m}n}} 
   \braket{\swY{\minus 2}{\ell^\prime{m^\prime}}{\theta,\phi}}{\swSH{\minus 2}{\ell{m}n}{\theta,\phi}{c_{\ell{m}n}}} \\
& {}\hspace{0.5in} 
  + A^\prime_{\ell{m}n} e^{i\omega_{\ell{m}n}t+i\phi^\prime_{\ell{m}n}}e^{-t/\tau_{\ell{m}n}} 
   \braket{\swY{\minus 2}{\ell^\prime{m^\prime}}{\theta,\phi}}{\swSH[*]{\minus 2}{\ell{m}n}{\pi-\theta,\phi}{c_{\ell{m}n}}} \Big\}.
\end{aligned}
\end{equation}

\noindent
Finally, using the expansion in Eqs.~(\ref{eqn:swSH:expan}) and (\ref{eqn:swSconj:expan}), and
the orthonormality of the spin-weighted spherical harmonics we get
\begin{equation}
\begin{aligned} \label{clm:eq3}
C_{\ell^\prime{m}^\prime}(t) & = \sum_{\ell{m}n} \Big\{ 
   A_{\ell{m}n} e^{-i\omega_{\ell{m}n}t+i\phi_{\ell{m}n}}e^{-t/\tau_{\ell{m}n}}
    \sum_{\ell^{\prime\prime}} \YSH{\ell^{\prime\prime}\ell{m}n}{c_{\ell{m}n}}
    \braket{\swY{\minus 2}{\ell^\prime{m}^\prime}{\theta,\phi}}{\swY{\minus 2}{\ell^{\prime\prime}m}{\theta,\phi}} \\
& {} \hspace{0.6in}
  + A^\prime_{\ell{m}n} e^{i\omega_{\ell{m}n}t+i\phi^\prime_{\ell{m}n}}e^{-t/\tau_{\ell{m}n}}
    \sum_{\ell^{\prime\prime}} (-1)^{\ell^{\prime\prime}} \YSH[*]{\ell^{\prime\prime}\ell{m}n}{c_{\ell{m}n}}
    \braket{\swY{\minus 2}{\ell^\prime{m}^\prime}{\theta,\phi}}{\swY{\minus 2}{\ell^{\prime\prime}(-m)}{\theta,\phi}} \Big\} \\ 
 & = \sum_{\ell{n}} \Big\{ 
   A_{\ell{m^\prime}n} e^{-i\omega_{\ell{m^\prime}n}t+i\phi_{\ell{m^\prime}n}}e^{-t/\tau_{\ell{m^\prime}n}}
    \YSH{\ell^\prime\ell{m^\prime}n}{c_{\ell{m^\prime}n}} \\
& {} \hspace{0.6in}
  + (-1)^{\ell^\prime}A^\prime_{\ell(-m^\prime)n} e^{i\omega_{\ell(-m^\prime)n}t+i\phi^\prime_{\ell(-m^\prime)n}}e^{-t/\tau_{\ell(-m^\prime)n}}
    \YSH[*]{\ell^\prime\ell(-m^\prime)n}{c_{\ell(-m^\prime)n}} \Big\}.
\end{aligned}
\end{equation}
After relabeling indices we have
\begin{equation}
\begin{aligned}
 C_{\ell{m}}(t) & = \sum_{\ell^\prime{n}} \Big\{ 
   A_{\ell^\prime{m}n}\YSH{\ell\ell^\prime{m}n}{c_{\ell^\prime{m}n}}
    e^{-i\omega_{\ell^\prime{m}n}t+i\phi_{\ell^\prime{m}n}}e^{-t/\tau_{\ell^\prime{m}n}} \\
& {} \hspace{0.6in}
  + (-1)^\ell A^\prime_{\ell^\prime(-m)n}
    \YSH[*]{\ell\ell^\prime(-m)n}{c_{\ell^\prime(-m)n}}
    e^{i\omega_{\ell^\prime(-m)n}t+i\phi^\prime_{\ell^\prime(-m)n}}e^{-t/\tau_{\ell^\prime(-m)n}} \Big\}.
\end{aligned}
\end{equation}
Finally, let us define two new functions:
\begin{align}
  \mathcal{R}_{\ell\ell^\prime{m}n}(t,\phi) &\equiv 
    \YSH{\ell\ell^\prime{m}n}{c_{\ell^\prime{m}n}}
    e^{-i\omega_{\ell^\prime{m}n}t+i\phi_{\ell^\prime{m}n}}e^{-t/\tau_{\ell^\prime{m}n}}, \\
  \mathcal{R}^\prime_{\ell\ell^\prime{m}n}(t,\phi) &\equiv  
    (-1)^\ell\YSH[*]{\ell\ell^\prime{m}n}{c_{\ell^\prime{m}n}}
    e^{i\omega_{\ell^\prime{m}n}t+i\phi{\ell^\prime{m}n}}e^{-t/\tau_{\ell^\prime{m}n}}.
\end{align}
So,
\begin{equation}
 C_{\ell{m}}(t) = \sum_{\ell^\prime{n}} \left\{ 
   A_{\ell^\prime{m}n}\mathcal{R}_{\ell\ell^\prime{m}n}(t,\phi_{\ell^\prime{m}n})
  + A^\prime_{\ell^\prime(-m)n}
     \mathcal{R}^\prime_{\ell\ell^\prime(-m)n}(t,\phi^\prime_{\ell^\prime(-m)n})
     \right\}.
\end{equation}
Thus, for each $\ell$ and $m$ we have four numbers $A_{\ell^\prime{m}n}$, $A^\prime_{\ell^\prime(-m)n}$, $\phi_{\ell^\prime{m}n}$, and $\phi^\prime_{\ell^\prime(-m)n}$ describing the wave, which later will be exploited as fit parameters.

\newpage
\noindent{\large\bf Appendix: Spin-Weighted Spheroidal Harmonics}
\vspace{0.25in}

The spin-weighted spheroidal harmonics,
$\swSH{s}{\ell{m}}{\theta,\phi}{c}$, are generalizations of the
spin-weighted spherical harmonics,
$\swY{s}{\ell{m}}{\theta,\phi}=\swSH{s}{\ell{m}}{\theta,\phi}{0}$,
where $s$ is the spin weight of the harmonic, and $c$ is the
oblateness parameter.  The angular dependence separates as
\begin{equation}
  \swSH{s}{\ell{m}}{\theta,\phi}{c} \equiv 
  \frac1{2\pi}\,\swS{s}{\ell{m}}{\cos\theta}{c}e^{im\phi}.
\end{equation}
With $x\equiv\cos\theta$, the spin-weighted spheroidal {\em function},
$\swS{s}{\ell{m}}{x}{c}$, satisfies
\begin{equation}\label{eqn:swSF_DiffEqn}
\partial_x \Big[ (1-x^2)\partial_x [\swS{s}{\ell{m}}{x}{c}]\Big] 
    + \bigg[(cx)^2 - 2 csx + s + \scA{s}{\ell m}{c} 
      - \frac{(m+sx)^2}{1-x^2}\bigg]\swS{s}{\ell{m}}{x}{c} = 0,
\end{equation}
where $\scA{s}{\ell m}{c}$ is the separation constant.

The basic symmetries inherent in the spin-weighted spheroidal harmonics
follow from~(\ref{eqn:swSF_DiffEqn}):
\begin{eqnarray}
s\rightarrow-s;\ x\rightarrow-x &\Rightarrow&
    \left\{\begin{array}{c}\swS{-s}{\ell{m}}{x}{c}=\pm\swS{s}{\ell{m}}{-x}{c} \\
        \scA{-s}{\ell{m}}{c} = \scA{s}{\ell{m}}{c} + 2s \end{array}\right. \\
m\rightarrow-m;\ x\rightarrow-x;\ c\rightarrow-c &\Rightarrow&
    \left\{\begin{array}{c}\swS{s}{\ell(-m)}{x}{c}=\pm\swS{s}{\ell{m}}{-x}{-c}\\
        \scA{s}{\ell(-m)}{c} = \scA{s}{\ell{m}}{-c} \end{array}\right. \\
\mbox{complex conjugation} &\Rightarrow&
    \left\{\begin{array}{c} \swS[*]{s}{\ell{m}}{x}{c}=\pm\swS{s}{\ell{m}}{x}{c^*} \\
        \scA[*]{s}{\ell{m}}{c} = \scA{s}{\ell{m}}{c^*} \end{array}\right.
\end{eqnarray}

Additional sign conventions are chosen for consistency with common
sign conventions for the angular-spheroidal functions,
$\swS{}{\ell{m}}{x}{c} = \swS{0}{\ell{m}}{x}{c}$, and the spin-weighted spheroidal harmonics, $\swY{s}{\ell{m}}{\theta,\phi}$.

\vspace{0.25in}
\noindent
{\it Angular spheroidal functions}
\begin{equation}
\partial_x \Big[ (1-x^2)\partial_x \swS{}{\ell{m}}{x}{c}\Big] + \bigg[(cx)^2 + \scA{}{\ell{m}}{c} - \frac{m^2}{1-x^2}\bigg]\swS{}{\ell{m}}{x}{c} = 0
\end{equation}
\begin{align}
\swS{}{\ell{m}}{-x}{c} &= (-1)^{\ell+m}\swS{}{\ell{m}}{x}{c}& \\
\swS{}{\ell{m}}{x}{-c} &= \swS{}{\ell{m}}{x}{c} 
           &:\ \scA{}{\ell{m}}{-c} = \scA{}{\ell{m}}{c} \\
\swS{}{\ell(-m)}{x}{c} &= (-1)^m\swS{}{\ell{m}}{x}{c} 
           &:\ \scA{}{\ell(-m)}{c} = \scA{}{\ell{m}}{c} \\
\swS[*]{}{\ell{m}}{x}{c} &= \swS{}{\ell{m}}{x}{c^*} 
           &:\ \scA[*]{}{\ell{m}}{c} = \scA{}{\ell{m}}{c^*}
\end{align}

\noindent
{\it Spin-weighted spherical functions}
\begin{equation}
\partial_x \Big[ (1-x^2)\partial_x [\swS{s}{\ell{m}}{x}{0}]\Big] + \bigg[s + \scA{s}{\ell{m}}{0} - \frac{(m+sx)^2}{1-x^2}\bigg]\swS{s}{\ell{m}}{x}{0} = 0
\end{equation}
\begin{equation}
\scA{s}{\ell{m}}{0} = l(l+1) - s(s+1)\\
\end{equation}
\begin{align}
\swS{-s}{\ell{m}}{x}{0} &= (-1)^{\ell+m}\swS{s}{\ell{m}}{-x}{0} 
           &:\ \scA{-s}{\ell{m}}{0} = \scA{s}{\ell{m}}{0} + 2s \\
\swS{s}{\ell(-m)}{x}{0} &= (-1)^{\ell+s}\swS{s}{\ell{m}}{-x}{0} 
           &:\ \scA{s}{\ell(-m)}{0} = \scA{s}{\ell{m}}{0} \\
\swS[*]{s}{\ell{m}}{x}{0} &= \swS{s}{\ell{m}}{x}{0} 
           &:\ \scA[*]{s}{\ell{m}}{0} = \scA{s}{\ell{m}}{0}
\end{align}

\noindent
{\it Spin-weighted spherical harmonics}
\begin{align}\label{eqn:swYminuss}
\swY{-s}{\ell{m}}{\theta,\phi}=(-1)^{\ell+m}\swY{s}{\ell{m}}{\pi-\theta,\phi} \\ \label{eqn:swYminusm}
\swY{s}{\ell(-m)}{\theta,\phi}=(-1)^{s+m}\swY[*]{-s}{\ell{m}}{\theta,\phi}
\end{align}

\noindent
These leave us with the following conventions for the spin-weighted spherical harmonics and spheroidal functions:
\begin{align}
\swS{-s}{\ell{m}}{x}{c} &= (-1)^{\ell+m}\swS{s}{\ell{m}}{-x}{c} 
           &:\ \scA{-s}{\ell{m}}{c} = \scA{s}{\ell{m}}{c} + 2s \\
\swS{s}{\ell(-m)}{x}{c} &= (-1)^{\ell+s}\swS{s}{\ell{m}}{-x}{-c} 
           &:\ \scA{s}{\ell(-m)}{c} = \scA{s}{\ell{m}}{-c} \\
\swS[*]{s}{\ell{m}}{x}{c} &= \swS{s}{\ell{m}}{x}{c^*} 
           &:\ \scA[*]{s}{\ell{m}}{c} = \scA{s}{\ell{m}}{c^*}
\end{align}
\begin{align}\label{eqn:swSHminuss}
\swSH{-s}{\ell{m}}{\theta,\phi}{c} &= (-1)^{\ell+m}\swSH{s}{\ell{m}}{\pi-\theta,\phi}{c} \\ \label{eqn:swSHminusm}
\swSH{s}{\ell(-m)}{\theta,\phi}{c} &= (-1)^{s+m}\swSH[*]{-s}{\ell{m}}{\theta,\phi}{-c^*}
\end{align}

\end{document}
